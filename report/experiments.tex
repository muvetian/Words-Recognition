
\section{Experiments}
\label{sec:expts}
It should be noted that in the original data set if every pixel grey scale value . Having a huge set of features may cause a commnon phenomena "the curse of dimensionality". This means that 
In order to reduce the complexity of the final model in the hope of avoiding overfitting problems, a feature selection process was implemented for the experiments.
\subsection{Feature Selections}
\label{feature}
According to Hall, feature selection process consists of the following steps:
\begin{itemize}
	\item Starting point
	\item Search orgnization
	\item Evaluation strategy
	\item Stopping criterion
In this section, the 
\end{itemize}

In the experiments
The backward elemination method was adpoted, which means that only deletions were considered.
\subsection{Classifiers}
\label{class}

were the questions you were trying to answer? What was the
experimental setup (number of trials, parameter settings, etc.)? What
were you measuring? You should justify these choices when
necessary. The accepted wisdom is that there should be enough detail
in this section that I could reproduce your work \emph{exactly} if I
were so motivated.
