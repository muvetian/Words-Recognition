
% The \section{} command formats and sets the title of this
% section. We'll deal with labels later.
\section{Introduction}
\label{sec:intro}

For this assignment we were given a MNIST dataset which consisted of 60,000
handwritten letters that each were converted into 28 x 28 set of
pixels displaying the color of each pixel. There was then a label for
each letter that we were then asked to predict for a test set. We
investigated different classifiers and looked at many resources to
determine what the greatest predictors were. 
\cite{milgram}

% Citations: As you can see above, you create a citation by using the
% \cite{} command. Inside the braces, you provide a "key" that is
% uniue to the paper/book/resource you are citing. How do you
% associate a key with a specific paper? You do so in a separate bib
% file --- for this document, the bib file is called
% project1.bib. Open that file to continue reading...

% Note that merely hitting the "return" key will not start a new line
% in LaTeX. To break a line, you need to end it with \\. To begin a 
% new paragraph, end a line with \\, leave a blank
% line, and then start the next line (like in this example).


